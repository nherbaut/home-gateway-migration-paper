We proposed a solution to help NFV becoming a reality for modular HG, making a first step to a fully virtualized home environment as proposed by ETSI.
We implemented a vNF and we used it to bring major enhancements to the HG, thanks to a small OSGi bundle.

More work should be carried out to optimize server-side task scheduling by using user's context information from the HG.
This would help giving greater priority to the most useful ones.
For OSGi modules, security, service isolation and management aspects has not been studied thoroughly and should be investigated.
Other network functions play a key role in vNF HG like NAT and should also be investigated.
For that, we must solve different issues, more related to network performances than to computational power. 
